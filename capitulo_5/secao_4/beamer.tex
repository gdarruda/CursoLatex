\documentclass[compress]{beamer}

\usetheme{Marburg} %Tema dos slides
\setbeamertemplate{navigation symbols}{} %Remove a barra de navegacao(bugada) do Beamer.


%----------Configuracoes e importacoes de pacotes----------%
\usepackage[T1]{fontenc}		 %Selecao de comentários no slide.
\usepackage[brazil]{babel}
\usepackage[utf8]{inputenc}  %Escrever acentos de forma trivial

\begin{document}

\title{Protocolo Revisão Sistemática}
\author{Orientando: Adilson Lopes Khouri}
\date{\today}
\institute[USP]{Orientador: Prof. Dr. Luciano Antonio Digiampietri}

\titlegraphic{\includegraphics[width=2.5cm]{./logo_each}} %Logo EACH

\begin{frame}
\titlepage
\end{frame}


% Sumário
\section*{Sumário}
\begin{frame}{Sumário}
\tableofcontents
\end{frame}

\section{Questões de Pesquisa}
\begin{frame}{}

\begin{block}{Questões de Pesquisa}
\begin{enumerate}
\item Quais são os métodos e as técnicas existentes para recomendar atividades em \emph{workf\mbox{}lows}?
\item Quais técnicas são utilizadas para validar os resultados obtidos?
\item Em quais áreas de aplicação as técnicas estão sendo utilizadas?
\end{enumerate}
\end{block}
\end{frame}

\section{Critérios de Inclusão e Exclusão}
\begin{frame}{}

\begin{block}{Critérios de Inclusão e Exclusão}
\textcolor{green}{Serão incluídos} os trabalhos que: Utilizem técnicas de recomendação de atividades.

\textcolor{red}{Serão excluídos} trabalhos que não são disponibilizados na íntegra e trabalhos que não descrevem a técnica. utilizada.
\end{block}

\begin{block}{Processo de Extração de Informação}
Após ler todos os abstracts dos artigos, serão aplicados os critérios de inclusão e exclusão em todos. 

Dos artigos incluídos serão extraídas as seguintes informações: nome, técnica aplicada para recomendar, se foi realizada alguma validação, em caso afirmativo, o tipo de validação realizado e o resultado obtido pela validação. 
\end{block}

\end{frame}

\end{document}