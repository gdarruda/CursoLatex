\documentclass[12pt, a4paper]{article}

\usepackage[utf8]{inputenc}

%	Pacote responsável por acrescentar 
%tabelas grandes em latex.
\usepackage{longtable}

%Início do documento em Latex
\begin{document}

\section{Tabelas}
Como fazer tabelas em \LaTeX\ (Comando para escrever a palavra \emph{Latex}) ?										\\
Como tudo no latex $\ldots$ será que existe um pacote para isso? 		\\
Tabelas com no máximo $1$ página e tabelas com mais de uma página. 	

\newpage
%	Exemplo de subsection, a numeracao é acoplada com a da section superior.
\subsection{Tabela com até uma página}
A Tabela \ref{ROTULO_01} (exemplo de referência cruzada) tem no máximo uma página.

%	Abre ambiente para tabelas com legenda e rótulo.
\begin{table}[h]

%	Centraliza a tabela
\center

%	Abre ambiente para tabelas. Os comandos {|l|r|c|} alinham 
%o contéudo das colunas a esquerda, direita e centro respectivamente.
\begin{tabular}{|l|r|c|}

 %	Desenha uma linha na tabela.
 \hline

 %	Título das colunas em negrito
 \textbf{Instâncias} & \textbf{Instâncias sem mising}  & \textbf{Características} \\ \hline
 
 %	Dados da tabela 
 569 & 569 & 30    \\ \hline

%	Fecha ambiente para tabelas.
\end{tabular}

%	Legenda
\caption{Exemplo de legenda.}

%	Rótulo para referencia cruzada
\label{ROTULO_01}

%	Fecha ambiente para tabelas com legenda e rótulo.
\end{table}

%	Força conteúdo para próxima página.
\newpage
\subsection{Tabela com mais de uma página}
Tabela com mais de uma página.

%	Abre ambiente para tabelas com mais de uma página.
\begin{longtable}{|c|l|l|l|}

%	Legenda e rótulo.
\caption{Exemplo de Legenda.}\label{ROTULO_02}
\\ \hline

%	Cabeçalho da primeira paǵina.
\multicolumn{1}{|l|}{\textbf{Artigo}} & 
\multicolumn{1}{l|}{\textbf{Inclusão}} & 
\multicolumn{1}{l|}{\textbf{Exclusão}} &
\multicolumn{1}{l|}{\textbf{Status}}\\ \hline \endfirsthead

%	Cabeçalho repetido em todas as páginas.
\multicolumn{4}{c}
{{\bfseries \tablename\ \thetable{} -- Continuação da página anterior}} \\
\hline \multicolumn{1}{|l|}{\textbf{Artigo}} &
\multicolumn{1}{l|}{\textbf{Inclusão}} &
\multicolumn{1}{l|}{\textbf{Exclusão}} &
\multicolumn{1}{|l|}{\textbf{Status}} \\ \hline
\endhead

%	Rodapé repetido em todas as páginas.
\hline \multicolumn{4}{|r|}{{Continua na próxima página}} \\ \hline
\endfoot
\endlastfoot

		%	Dados da tabela
	   	1	&	A(X)	&	B e C	& excluído	\\  \hline
		2	&	A(X)	&	B e C	& excluído	\\  \hline
		3	&	A(X)&	B e C	& excluído	\\  \hline
		4	&	A   &	B e C	& incluído	\\  \hline
		5	&	A(X)&	B e C	& excluído	\\  \hline
		6	&	A   &	B e C	& incluído	\\  \hline
		7	&	A(X)&	B e C	& excluído	\\  \hline
		8	&	A(X)&	B e C	& excluído	\\  \hline
		9	&	A(X)&	B e C	& excluído	\\  \hline
		10	&	A(X)&	B e C	& excluído	\\  	\hline
		11	&	A(X)	&	B e C	& excluído	\\  \hline
		12	&	A	&	B e C	& incluído	\\  \hline
		13	&	A	&	B e C	& incluído	\\  \hline
		14	&	A(X)&	B e C	& excluído	\\  	\hline
		15	&	A	&	B e C	& incluído	\\  \hline
		16	&	A(X)&	B e C	& excluído	\\  \hline
		17	&	A(X)&	B e C	& excluído	\\  \hline
		18	&	A	&	B e C	& incluído	\\  \hline
		19	&	A(X)&	B e C	& excluído	\\  \hline
		20	&	A(X)&	B e C	& excluído	\\  \hline
		21	&	A	&	B e C	& incluído	\\  \hline
		22	&	A(X)&	B e C	& excluído	\\  \hline
		23	&	A(X)&	B e C	& excluído	\\  \hline
		24	&	A	&	B e C	& incluído	\\  \hline
		25	&	A	&	B e C	& incluído	\\  \hline
		26	&	A(X)&	B e C	& excluído	\\  \hline
		27	&	A(X)&	B e C	& excluído	\\  \hline
		28	&	A	&	B e C	& incluído	\\  \hline
		29	&	A	&	B e C	& incluído	\\  \hline
		30	&	A(X)	&	B e C	& excluído	\\  \hline
		31	&	A(X)	&	B e C	& excluído	\\  \hline
		32	&	A	&	B e C	& incluído	\\  \hline
		33	&	A(X)&	B e C	& excluído	\\  \hline
		34	&	A(X)&	B e C	& excluído	\\  \hline
		35	&	A	&	B e C	& incluído 	\\  \hline
		36	&	A(X)&	B e C	& excluído 	\\  \hline
		37	&	A(X)&	B e C	& excluído	\\  \hline
		38	&	A(X)&	B e C	& excluído	\\  \hline
		39	&	A(X)&	B e C	& excluído	\\  \hline
		40	&	A	&	B e C	& incluído	\\  \hline
		41	&	A	&	B e C	& incluído	\\  \hline
		42	&	A	&	B e C	& incluído	\\  \hline
		43	&	A(X)&	B e C	& excluído	\\  \hline
		44	&	A	&	B e C	& incluído	\\  \hline
		45	&	A	&	B e C	& incluído	\\  \hline
		46	&	A	&	B e C	& incluído	\\  \hline
		47	&	A(X)&	B e C	& excluído	\\  \hline
		48	&	A(X)&	B e C	& excluído	\\  \hline
		49	&	A(X)&	B e C	& excluído	\\  \hline
		50	&	A	&	B e C	& incluído	\\  \hline
		51	&	A(X)&	B e C	& excluído	\\  \hline
		52	&	A	&	B e C	& incluído	\\	\hline
		53	&	A(X)	&	B e C	& excluído	\\  \hline
		54	&	A(X)	&	B e C	& excluído	\\  \hline
		55	&	A(X)&	B e C	& excluído	\\  \hline
		56	&	A	&	B e C	& incluído	\\  \hline
		57	&	A(X)&	B e C	& excluído	\\  \hline
		58	&	A(X)&	B e C	& excluído	\\  \hline
		59	&	A(X)&	B e C	& excluído	\\  \hline
		60	&	A(X)&	B e C	& excluído	\\  \hline
		61	&	A(X)	&	B e C	& excluído	\\  \hline
		62	&	A(X)	&	B e C	& excluído	\\  \hline
		63	&	A(X)&	B e C	& excluído	\\  \hline
		64	&	A(X)&	B e C	& excluído	\\  \hline
		65	&	A(X)&	B e C	& excluído	\\  \hline
		66	&	A(X)&	B e C	& excluído	\\  \hline
		67	&	A(X)&	B e C	& excluído	\\  \hline
		68	&	A(X)&	B e C	& excluído	\\  \hline
		69	&	A(X)&	B e C	& excluído	\\  \hline
		70	&	A(X)&	B e C	& excluído	\\  	\hline		
		71	&	A(X)	&	B e C	& excluído	\\  	\hline		
		72	&	A(X)&	B e C	& excluído	\\  \hline
		73	&	A(X)&	B e C	& excluído	\\  \hline
		74	&	A(X)&	B e C	& excluído	\\  \hline
		75	&	A(X)&	B e C	& excluído	\\  \hline
		76	&	A(X)&	B e C	& excluído	\\  \hline
		77	&	A(X)&	B e C	& excluído	\\ \hline

%	Fecha ambiente para tabelas com mais de uma página.
\end{longtable}

%	Fim do documento em Latex
\end{document}