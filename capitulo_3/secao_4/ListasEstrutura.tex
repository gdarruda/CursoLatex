\documentclass[12pt, a4paper]{article}

\usepackage[utf8]{inputenc}

%	Início do documento em Latex
\begin{document}

%	Abre ambiente de section.
\section{Listas}
Como fazer listas em \LaTeX\ (Comando para escrever a palavra \emph{Latex}) ?	\\
Como tudo no latex $\ldots$ será que existe um pacote para isso? 	

%	Abre o ambiente subsection subordinado ao ambiente \section{Listas}.
\subsection{Itemize}
%	Abre o ambiente itemize que lista objetos 
%com bolinhas pretas ao lado (padrão[\item item a]) ou um 
%símbolo customizável (\item[$\beta$] item b)
\begin{itemize}
\item item a
\item[$\beta$] item b
\item[$\gamma$] item c
\end{itemize}


%	Abre o ambiente subsubsection subordinado ao ambiente \subsection{Itemize}.
\subsubsection{Comentários}
As listas são úteis.

%	Esta subsection é subordinada ao ambiente \section{Listas}.
\subsection{Enumerate}

%	Abre o ambiente enumerate que enumera 
%objetos de 1 a n, onde n é o número de objetos.
\begin{enumerate}
\item com número
\item com número
\item com número
\end{enumerate}

\subsection{Lista encadeadas}
%	Exemplo de listas encadeadas.
\begin{itemize}
    \item Primeiro nível, \emph{itemize}, primeiro item.
    
    %	Onde viria o próximo item abrimos um novo ambiente itemize
    %dessa forma, criamos o encadeamento de segundo nível.
    \begin{itemize}
        \item Segundo nível, \emph{itemize}, primeiro item.
        \item Segundo nível, \emph{itemize}, segundo item.
        
    		%	Onde viria o próximo item abrimos um ambiente enumerate
    		%dessa forma, criamos o encadeamento de terceiro nível.
        \begin{enumerate}
            \item Terceiro nível, \emph{enumerate}, primeiro item.
            \item Terceiro nível, \emph{enumerate}, segundo item.

		%	Fecha o ambiente de terceiro nível.
        \end{enumerate}
        
    %	Fecha o ambiente de segundo nível.
    \end{itemize}
    \item Primeiro nível, \emph{itemize}, segundo item.
\end{itemize}



%	Fim do documento em Latex
\end{document}