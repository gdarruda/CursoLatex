\documentclass[12pt, a4paper]{article}

\usepackage[utf8]{inputenc}

%	Adiciona pacotes de cores em latex, 
%necessário para colorir o código.
\usepackage{color}
\usepackage{xcolor}  

%	Configura a exibição do código 
%fonte dentro do ambiente listing.
\usepackage{listings}
\lstset{		
	language = R, 					% Linguagem de programação.
	keywordstyle = \color{blue}, 		% Cor das palavras chaves.
	commentstyle = \color{green}, 	% Cor dos Comentários.
	stringstyle =  \color{red}, 		% Cor de Strings.
	numbers=left,					% Posição da numeração das linhas.
	stepnumber=1,					% Incremento dos números a cada x unidades.
	firstnumber=1,					% Começa a numeração em x.
	numberstyle=\tiny,				% Tamanho dos números.
	extendedchars=true,				% Aceita caracteres extras.
	breaklines=true,					% Aceita quebra de linhas.
	frame=tb,						% Acrescenta uma caixa em cima e embaixo do código.
	basicstyle=\footnotesize,		% Tamanho do código.
	showstringspaces=false,			% Exibe espaço entre strings.
    }
        
        
%	Início do documento em Latex
\begin{document}

\section{Código Fonte}
Vamos apresentar uma forma elegante para incluir código fonte no seu arquivo \LaTeX\. Primeiro é necessário um arquivo fonte de alguma linguagem de programação e.g. C, VB, R

%	Entre colchetes[] alguns parâmetros: rótulo, linguagem e legenda respectivamente.
%	Entre chaves {} o caminho relativo do arquivo em ambiente UNIX like, 
%se estiver no Windows ALTERE a barra / para \
%-------------------------------------------------------------------------------------%
\lstinputlisting[label=CODIGO01,language=R,caption=Script geral que resolve a tarefa 01.]{./Tarefa01.R}
%-------------------------------------------------------------------------------------%


%	Fim do documento em Latex
\end{document}