\documentclass[12pt, a4paper]{article}

\usepackage[utf8]{inputenc}

%Pacote para tratar figuras
\usepackage{graphicx}
\usepackage{subfig}


%	Início do documento em Latex
\begin{document}

\section{Figuras}
Como inserir figuras em \LaTeX\ (Comando para escrever a palavra \emph{Latex}) ? \\
Como tudo no latex $\ldots$ será que existe um pacote para isso? 		\\
Figuras únicas e lado a lado.

\section{Figura única}
%	Abre ambiente para adicionar figura única.
\begin{figure}[!htb]
	%	Centraliza a figura
    \centering
    
    %	Legenda na figura
    \caption{Figura única.}
	
	%	Entre [] a scala da figura e entre {} o caminho relativo
	%em ambiente UNIX like se for Windows trocar a barra para \
    \includegraphics[scale=0.4]{./kenny}
	
	%	Rótulo para figura
    \label{grafico_car}
\end{figure}

\newpage

\section{Figuras lado a lado}
%	Abre ambiente para adicionar figura.
\begin{figure}[!htb]

%	Centraliza a figura
\centering

%	Cria subfigura interna 01 entre [] legenda: cartman e heigth (altura)
%	Entre {} caminho relativo
\subfloat[cartman]{
\includegraphics[height=5cm]{cartman}

%Rótulo da subfigura cartman
\label{cartman}
}

%	Espaco entre subfiguras
\quad 

%	Cria subfigura interna 02 entre [] legenda: kenny e heigth (altura)
%	Entre {} caminho relativo
\subfloat[kenny]{

%	Cria subfigura interna 01 entre [] legenda: cartman e heigth (altura)
%	Entre {} caminho relativo
\includegraphics[height=5cm]{kenny}

%Rótulo da subfigura cartman
\label{kenny}
}

%Legenda e Rótulo das duas figuras juntas (cartman + kenny)
\caption{Subfiguras lado a lado}
\label{fig01}
\end{figure}

\newpage

\section{Nota sobre os parâmetros [!htb]}
Nos parâmetros opcionais das figuras (aqueles entre colchetes) há a opção $!htb$ sua função é de alinhar a figura na página, seu significado é
\begin{itemize}
\item h, here
\item t, top
\item b, bottom
\item p, page of float
\end{itemize}


%	Fim do documento em Latex
\end{document}