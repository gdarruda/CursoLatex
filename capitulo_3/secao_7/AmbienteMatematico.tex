\documentclass[12pt, a4paper]{article}

\usepackage[utf8]{inputenc}


%Pacote para ambiente matemático
\usepackage{amsmath} 
\usepackage{amssymb} 
\usepackage{mathtools}

%Verbatim
\usepackage{verbatim}

%	Início do documento em Latex
\begin{document}
Como fazer equações em \LaTeX\ (Comando para escrever a palavra \emph{Latex}) ?\\
Como tudo no latex $\ldots$ será que existe um pacote para isso? 		\\
Equações em linha e em bloco.

\section{Equações em Linha}
Equações pequenas podem ser escritas dentro do texto $x = x_{0} + e^{\sqrt{x!}}\frac{\pi}{\alpha}$, em outras palavras, continua na mesma linha.

\section{Equações em Bloco}
Para equações maiores podemos utilizar o ambiente em bloco.

%	Abre ambiente matemático em bloco no latex
%para trocar de linha usamos \\ o comando \nonumber 
%evita que uma linha seja numerada
\begin{eqnarray}
\nonumber 5n^{4} - 37n^{3} + 13n - 4 &\leq& Kn^{4} \Leftrightarrow \\
5n^{4} - 37n^{3} + 13n - 4 &\leq& 5n^{4} - 37n^{4} + 13n^{4} -4n^{4} \leq  \\
\nonumber -23n^{4} &\leq&  Kn^{4}
\end{eqnarray}

\section{Alguns símbolos matemáticos}
\begin{itemize}
\item menor ou igual $\leq$.
\item para elevar um número ao qudrado $x^{2}$ usamos o circunflexo.
\item para escrever um índice em uma variável usamos o \emph{underscore} $x_{2}$.
\item para obter a raiz cúbica de um número usamos o comando $\sqrt[3]{x}$.
\item para obter a razão entre $a$ e $b$ usamos o comando $\frac{a}{b}$.
\item há letras gregas e número famosos e.g. $\pi$, $\alpha$, $\Omega$.
\end{itemize}


%	Fim do documento em Latex
\end{document}