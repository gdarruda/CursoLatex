\documentclass[a4paper]{abntex2}
%ou \documentclass[a4paper]{abnt} para versão antiga desse pacote

\usepackage[utf8]{inputenc}

%Pacote para citação abnt.
\usepackage[alf]{abntcite}

%	Dica: para o Bibtex funcionar devemos primeiro compilar o documento com 
%pdflatex, depois com a ferramenta Bibtex e por fim, novamente com a ferramenta pdflatex.

%	Dica2: é possível montar um makefile ou .bat com essas chamadas de comandos (ver Dica 1) na ordem correta
%isso facilita a compilação

%	Início do documento em Latex
\begin{document}


\chapter{Como citar e fazer referências}
Para elaborar referências utilizaremos o bibtex e para citar o comando \emph{cite}. O primeiro passo é obter as referências em formato bibtex. Segundo \cite{Kira1992, Kononenko94} (referência indireta) xpto é verdade segundo \citeonline{Kononenko94} (referência direta).


%Elabora a bibliografia com os dados do arquivo bibtex.
\bibliography{referencias}

%	Fim do documento em Latex
\end{document}